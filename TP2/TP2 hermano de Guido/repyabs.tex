\documentclass{article}
\usepackage[paper=a4paper, left=1.5cm, right=1.5cm, bottom=1.5cm, top=3.5cm]{geometry}
\usepackage{fancyhdr}
	
\pagestyle{fancy}
\thispagestyle{fancy}
\addtolength{\headheight}{1pt}
\lhead{Trabajo Pr\'{a}ctico 2}
\rhead{Grupo 16}

\begin{document}
\section*{Representacion}


\hspace{1cm}1) Un tema esta en e.temas si y solo si existe un autor que lo contenga como tema suyo. 

\hspace{0.5cm}2) Un tema pertenece a e.temas si y solo si es clave de e.FTS.

\hspace{0.5cm}3) Los autores que no estan definidos en e.autores entonces nunca van a ser ultimoAutor en e.cacheoAutor

\hspace{0.5cm} y no son coautor de ningun autor.

\hspace{0.5cm}4) Toda clave de e.FTS pertenece a su propio significado.

\hspace{0.5cm}5) si un tema no tiene el mismo nombre que un autor entonces su significado es de tamaño 1

\hspace{0.5cm}6) si un tema tiene el mismo nombre que un autor y ademas el tema pertenece a ese autor entonces su significado

\hspace{0.5cm} en e.FTS es el mismo que en e.PorAutor 

\hspace{0.5cm}7) si un tema tiene el mismo nombre que un autor pero el tema NO pertenece a ese autor entonces su significado

\hspace{0.5cm} en e.FTS es el mismo que en e.PorAutor mas su propio nombre.

\hspace{0.5cm}8) Todos los autores que son clave de e.PorAutor pertenece a e.autores. Y para todas esos autores: 

\hspace{1cm} a) los temas que disputo pertenecen a e.temas

\hspace{1cm} b) los temas del autor que no fueron disputados pertenecen al conjunto de sus Temas 

\hspace{1cm} c) el autor no es su propio Coautor y todos sus Coautores pertenecen a e.autores y ademas son claves del

\hspace{1cm} diccionario e.PorAutor

\hspace{1cm} d) son coautores aquellos autores con los que tiene al menos un tema en comun.

\hspace{1cm} e) sus temas no pertenecen a los temas que disputo y viceversa 

\hspace{0.5cm}9)El autor de ultimoAutor en e.cacheoAutor es clave del diccionario e.porAutor

\hspace{0.5cm}10)En e.cacheoAutor el itInicio apunta al mismo conjunto que el iterador de temasCacheo
 
\hspace{0.5cm}11)Los temas que estan dentro del conjunto al que apunta el iterador son los temas que disputo el ultimoAutor
\\


\hspace{0.5cm} \textbf{Rep}: sadaic e $\rightarrow$ bool 

\hspace{0.5cm} Rep(e) $\equiv$

\hspace{1cm}1) ($\forall$ t:tema) t $\in$ e.temas $\Leftrightarrow$ (($\exists$ a:autor) a $\in$ claves(e.PorAutor) $\wedge_{L}$ ($\exists$ it:itConj(tema)) 

\hspace{1cm} it $\in$ (Significado(e.PorAutor,a).susTemas) $\wedge_{L}$ siguiente(it) = t) $\wedge$
\\

\hspace{1cm}2)($\forall$ t:tema) t $\in$ e.temas $\Leftrightarrow$ t $\in$ claves(e.FTS) $\wedge$ 
\\

\hspace{1cm}3)($\forall$ a:autor) $\neg$definido?(e.PorAutor,a) $\Rightarrow$ (a != (e.cacheoAutor).ultimoAutor $\wedge$ ($\not\exists$ a':autor) a' $\in$ claves(e.PorAutor)

\hspace{1cm}$\wedge_{L}$ a $\in$ (significado(e.PorAutor,a').susCoautores)) $\wedge$
\\

\hspace{1cm}4)($\forall$ t:tema) t $\in$ claves(e.FTS) $\Rightarrow$ ($\exists$ it:itConj(tema)) it $\in$ significado(e.FTS,t) $\wedge_{L}$ t = siguiente(it) $\wedge$
\\

\hspace{1cm}5)($\forall$ t:tema) t $\in$ claves(e.FTS) $\wedge$ t $\notin$ e.autores $\Rightarrow$ $\#$(significado(e.FTS,t)) = 1 $\wedge$
\\
 
\hspace{1cm}6)($\forall$ t:tema) t $\in$ claves(e.FTS) $\wedge$ t $\in$ e.autores $\wedge_{L}$ (($\exists$ it:itConj(tema)) it $\in$ significado(e.PorAutor,t).susTemas 

\hspace{1cm} $\wedge_{L}$ t=siguiente(it)) $\Leftrightarrow$ significado(e.FTS,t) = significado(e.PorAutor,t) $\wedge$
\\

\hspace{1cm}7)($\forall$ t:tema) t $\in$ claves(e.FTS) $\wedge$ t $\in$ e.autores $\wedge_{L}$ ($\not\exists$ it:itCOnj(tema)) it $\in$ significado(e.PorAutor,t).susTemas 

\hspace{1cm} $\wedge_{L}$ t=siguiente(it) $\Rightarrow$ ($\forall$ it':itConj(tema)) it' $\in$ significado(e.porAutor,t).susTemas $\Rightarrow$ it' $\in$ significado(e.FTS,t)

\hspace{1cm} $\wedge$  $\#$(significado(e.FTS,t)) = $\#$(significado(e.PorAutor,t).susTemas) + 1 $\wedge$
\\

\hspace{1cm}8)($\forall$ a:autor) a $\in$ claves(e.PorAutor) $\Rightarrow$ a $\in$ e.autores $\wedge$
\\

\hspace{1.2cm}a)($\forall$ it:itConj(tema)) it $\in$ significado(e.PorAutor,a).temasQDisputo 
$\Rightarrow$ siguiente(it) $\in$ e.temas)$\wedge$
\\

\hspace{1.2cm}b)($\forall$ it:itConj(tema)) it $\in$ significado(e.PorAutor).temasNosDisputados $\Rightarrow$ it $\in$ significado(e.PorAutor).susTemas$\wedge$
\\

\hspace{1.2cm}c)($\forall$ a':autor) a' $\in$ significado(e.PorAutor,a).susCoautores $\Leftrightarrow$ a' $\in$ e.autores $\wedge$ a' != a $\wedge$ a' $\in$ claves(e.PorAutor))$\wedge$
\\

\hspace{1.2cm}d)($\forall$ a':autor) a' $\in$ significado(e.PorAutor,a) $\Leftrightarrow$

\hspace{1.2cm}($\exists$ it:itConj(tema)) it $\in$ significado(e.PorAutor,a).susTemas $\wedge$ it $\in$ significado(e.PorAutor,a').susTemas $\wedge$
\\

\hspace{1.2cm}e)($\forall$ it:itConj(tema)) it $\in$ significado(e.PorAutor,a).susTemas $\Leftrightarrow$ 

\hspace{1.2cm} it $\notin$ significado(e.PorAutor,a).temasQDisputo $\wedge$
\\

\hspace{1cm}9)(e.cacheoAutor).ultimoAutor $\in$ clave(e.PorAutor) $\wedge$
\\

\hspace{1cm}10)Alias((e.cacheoAutor).itInicio = (e.cacheoAutor).temasCacheo) $\wedge$ 
\\

\hspace{1cm}11)($\forall$ it:itConj(tema)) esta?(it,secuSuby((e.cacheoAutor).temasCacheo)) $\Leftrightarrow$ 

\hspace{1cm}it $\in$ significado(e.PorAutor,(e.cacheoAutor).ultimoAutor).TemasQdisputo
\\

\hspace{1cm} \textbf{Abs}(e) $\equiv$ s:sadaic $|$

\hspace{1cm} temas(s) $=_{obs}$ e.temas $\wedge$ autores(s) $=_{obs}$ e.autores $\wedge$ ($\forall$ a:autor) a $\in$ autores(s) $\Rightarrow$ 

\hspace{1cm} (temasDe(s,a) $=_{obs}$ significado(e.PorAutor,a).susTemas 

\hspace{1cm} $\wedge$ temasDisputadosPor(s,a) $=_{obs}$ significado(e.PorAutor,a).temasQDisputo) 

\end{document}