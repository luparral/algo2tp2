\documentclass[10pt, a4paper]{article}
\usepackage[paper=a4paper, left=1.5cm, right=1.5cm, bottom=1.5cm, top=3.5cm]{geometry}
\usepackage[latin1]{inputenc}
\usepackage[T1]{fontenc}
\usepackage[spanish]{babel}
\usepackage{indentfirst}
\usepackage{fancyhdr}
\usepackage{latexsym}
\usepackage{lastpage}
\usepackage{aed2-symb,aed2-itef,aed2-tad}
\usepackage[colorlinks=true, linkcolor=blue]{hyperref}
\usepackage{calc}

\newcommand{\f}[1]{\text{#1}}
\renewcommand{\paratodo}[2]{$\forall~#2$: #1}

\sloppy

\hypersetup{%
 % Para que el PDF se abra a p�gina completa.
 pdfstartview= {FitH \hypercalcbp{\paperheight-\topmargin-1in-\headheight}},
 pdfauthor={C�tedra de Algoritmos y Estructuras de Datos II - DC - UBA},
 pdfkeywords={TADs b�sicos},
 pdfsubject={Tipos abstractos de datos b�sicos}
}

\parskip=5pt % 10pt es el tama�o de fuente

% Pongo en 0 la distancia extra entre �temes.
\let\olditemize\itemize
\def\itemize{\olditemize\itemsep=0pt}

% Acomodo fancyhdr.
\pagestyle{fancy}
\thispagestyle{fancy}
\addtolength{\headheight}{1pt}
\lhead{Algoritmos y Estructuras de Datos II}
\rhead{$2^{\mathrm{do}}$ cuatrimestre de 2013}
\cfoot{\thepage /\pageref{LastPage}}
\renewcommand{\footrulewidth}{0.4pt}

\author{Algoritmos y Estructuras de Datos II, DC, UBA.}
\date{}
\title{Tipos abstractos de datos b�sicos}

\begin{document}

\section{TAD \tadNombre{Diccionario Rapido($\alpha$)}}

\begin{tad}{\tadNombre{Diccionario Rapido($\alpha$)}}

\tadIgualdadObservacional{d}{d'}{dicc(string,$\alpha$)}{$(\forall s:string)$(definido?($s, d$) $\igobs$ definido?($s, d'$) $\land$\\ (definido?($c, d$) $\implies$ obtener($s, d$) $\igobs$ obtener($s, d'$)))}

\tadParametrosFormales{
    \tadEncabezadoInline{g�neros}{$string$, $\alpha$}
}

\tadGeneros{diccS(string,$\alpha$)}
\tadExporta{diccS(string,$\alpha$), generadores, observadores, borrar, claves}
\tadUsa{\tadNombre{Bool, Nat, Conjunto(String)}}

%notar la importancia de agrupar el ultimo parametro, para la coma
\tadAlinearFunciones{todosLosSignificados}{clave,significado,{dicc(clave, significado)}}

\tadObservadores 
\tadOperacion{definido?}{string}{bool}{} 
\tadOperacion{significado}{string/s,{diccS(string,$\alpha$)}}{$\alpha$}{definido?($s$, $d$)}

\tadGeneradores
\tadOperacion{crearDicc}{}{diccS(string,$\alpha$)}{}
\tadOperacion{definir}{string,$\alpha$,{diccS(string,$\alpha$)}}{diccS(string,$\alpha$)}{}

\tadOtrasOperaciones
\tadOperacion{borrar}{string/s,{diccS(string,$\alpha$)}/d}{diccS($\alpha$)}{definido?($s$,$d$)}
\tadOperacion{claves}{{diccS(string,$\alpha$)}}{conj(string)}{}
\tadOperacion{significadosHijos}{diccS(string,$\alpha$),string/s}{diccS($\alpha$)}{}
\tadOperacion{todosLosSignificados}{{diccS(string,$\alpha$)}}{conj($\alpha$)}{}


\tadAxiomas[\paratodo{diccS(string,$\alpha$)}{d}, \paratodo{string}{s, s'}, \paratodo{$\alpha$}{a}]
\tadAlinearAxiomas{todosLosSignificados($c$, definir($k$, $s$, $d$))}

\tadAxioma{definido?($s$, crearDicc)}{false}
\tadAxioma{definido?($s$, definir($s'$, $a$, $d$)}{$s = s'$ $\lor$ definido?($s$, $d$)}

\tadAxioma{significado($s$, definir($s'$, $a$, $d$))}{\IF $s = s'$ THEN $a$ ELSE significado($s$, $d$) FI}

\tadAxioma{borrar($s$, definir($s'$, $a$, $d$))}{\IF $s = s'$ THEN {\IF definido?($s$,$d$) THEN borrar($s$,$d$) ELSE $d$ FI} ELSE
definir($s'$, $a$, borrar($s$, $d$)) FI}

\tadAxioma{claves(crearDicc)}{$\emptyset$}
\tadAxioma{claves(definir($s$, $a$, $d$)}{Ag($s$, claves($d$))}

\tadAxioma{significadosHijos(crearDicc,s)}{$\emptyset$}
\tadAxioma{significadosHijos(definir($s'$,$a$,$d$),$s$)}{if coincide(s',s) then Ag(a, significadosHijos($d$,$s$) else significadosHijos($d$,$s$) fi}

\tadAxioma{todosLosSignificados(crearDicc)}{$\emptyset$}
\tadAxioma{todosLosSignificados(definir($s$, $a$, $d$)}{Ag($a$, todosLosSignificados($d$)}

\end{tad}

\end{document}
