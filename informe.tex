\documentclass[a4paper,10pt]{article}
\usepackage[paper=a4paper, left=1cm, right=0.9cm, bottom=1.5cm, top=3.5cm]{geometry}
\usepackage[latin1]{inputenc}
\usepackage[T1]{fontenc}
\usepackage[spanish]{babel}
\usepackage{indentfirst}
\usepackage{fancyhdr}
\usepackage{latexsym}
\usepackage{lastpage}
\usepackage{aed2-symb,aed2-itef,aed2-tad}
\usepackage[colorlinks=true, linkcolor=cyan]{hyperref}
\usepackage{calc}
\usepackage{caratula}
\usepackage{algorithm}% http://ctan.org/pkg/algorithms
\usepackage{algpseudocode}% http://ctan.org/pkg/algorithmicx
\usepackage{verbatim}



\parskip=5pt % 10pt es el tamaño de fuente

% Pongo en 0 la distancia extra entre ítemes.
\let\olditemize\itemize
\def\itemize{\olditemize\itemsep=0pt}

% Acomodo fancyhdr.
\pagestyle{fancy}
\thispagestyle{fancy}
\addtolength{\headheight}{1pt}
\lhead{Algoritmos y Estructuras de Datos II}
\rhead{$1^{\mathrm{er}}$ cuatrimestre de 2013}
\cfoot{\thepage /\pageref{LastPage}}
\renewcommand{\footrulewidth}{0.4pt}

\author{Algoritmos y Estructuras de Datos II, DC, UBA.}
\date{}
\title{Trabajo practico de especificacion}


\newcommand{\moduloNombre}[1]{\textbf{#1}}

\let\NombreFuncion=\textsc
\let\TipoVariable=\texttt
\let\ModificadorArgumento=\textbf
\newcommand{\res}{$res$\xspace}
\newcommand{\tab}{\hspace*{7mm}}

\newcommandx{\TipoFuncion}[3]{%
  \NombreFuncion{#1}(#2) \ifx#3\empty\else $\to$ \res\,: \TipoVariable{#3}\fi%
}
\newcommand{\In}[2]{\ModificadorArgumento{in} \ensuremath{#1}\,: \TipoVariable{#2}\xspace}
\newcommand{\Out}[2]{\ModificadorArgumento{out} \ensuremath{#1}\,: \TipoVariable{#2}\xspace}
\newcommand{\Inout}[2]{\ModificadorArgumento{in/out} \ensuremath{#1}\,: \TipoVariable{#2}\xspace}
\newcommand{\Aplicar}[2]{\NombreFuncion{#1}(#2)}

\newlength{\IntFuncionLengthA}
\newlength{\IntFuncionLengthB}
\newlength{\IntFuncionLengthC}
%InterfazFuncion(nombre, argumentos, valor retorno, precondicion, postcondicion, complejidad, descripcion, aliasing)
\newcommandx{\InterfazFuncion}[9][4=true,6,7,8,9]{%
  \hangindent=\parindent
  \TipoFuncion{#1}{#2}{#3}\\%
  \textbf{Pre} $\equiv$ \{#4\}\\%
  \textbf{Post} $\equiv$ \{#5\}%
  \ifx#6\empty\else\\\textbf{Complejidad:} #6\fi%
  \ifx#7\empty\else\\\textbf{Descripcion:} #7\fi%
  \ifx#8\empty\else\\\textbf{Aliasing:} #8\fi%
  \ifx#9\empty\else\\\textbf{Requiere:} #9\fi%
}

\newenvironment{Interfaz}{%
  \parskip=2ex%
  \noindent\textbf{\Large Interfaz}%
  \par%
}{}

\newenvironment{Representacion}{%
  \vspace*{2ex}%
  \noindent\textbf{\Large Representacion}%
  \vspace*{2ex}%
}{}

\newenvironment{Algoritmos}{%
  \vspace*{2ex}%
  \noindent\textbf{\Large Algoritmos}%
  \vspace*{2ex}%
}{}


\newcommand{\Titulo}[1]{
  \vspace*{1ex}\par\noindent\textbf{\large #1}\par
}

\newenvironmentx{Estructura}[2][2={estr}]{%
  \par\vspace*{2ex}%
  \TipoVariable{#1} \textbf{se representa con} \TipoVariable{#2}%
  \par\vspace*{1ex}%
}{%
  \par\vspace*{2ex}%
}%

\newboolean{EstructuraHayItems}
\newlength{\lenTupla}
\newenvironmentx{Tupla}[1][1={estr}]{%
    \settowidth{\lenTupla}{\hspace*{3mm}donde \TipoVariable{#1} es \TipoVariable{tupla}$($}%
    \addtolength{\lenTupla}{\parindent}%
    \hspace*{3mm}donde \TipoVariable{#1} es \TipoVariable{tupla}$($%
    \begin{minipage}[t]{\linewidth-\lenTupla}%
    \setboolean{EstructuraHayItems}{false}%
}{%
    $)$%
    \end{minipage}
}

\newcommandx{\tupItem}[3][1={\ }]{%
    %\hspace*{3mm}%
    \ifthenelse{\boolean{EstructuraHayItems}}{%
        #1%                                                --------------aca borre una coma antes del # (nico)
    }{}%
    \emph{#2}: \TipoVariable{#3}%
    \setboolean{EstructuraHayItems}{true}%
}

\newcommandx{\RepFc}[3][1={estr},2={e}]{%
  \tadOperacion{Rep}{#1}{bool}{}%
  \tadAxioma{Rep($#2$)}{#3}%
}%

\newcommandx{\Rep}[3][1={estr},2={e}]{%
  \tadOperacion{Rep}{#1}{bool}{}%
  \tadAxioma{Rep($#2$)}{true \ssi #3}%
}%

\newcommandx{\Abs}[5][1={estr},3={e}]{%
  \tadOperacion{Abs}{#1/#3}{#2}{Rep($#3$)}%
  \settominwidth{\hangindent}{Abs($#3$) \igobs #4: #2 $\mid$ }%
  \addtolength{\hangindent}{\parindent}%
  Abs($#3$) \igobs #4: #2 $\mid$ #5%
}%

\newcommandx{\AbsFc}[4][1={estr},3={e}]{%
  \tadOperacion{Abs}{#1/#3}{#2}{Rep($#3$)}%
  \tadAxioma{Abs($#3$)}{#4}%
}%


\newcommand{\DRef}{\ensuremath{\rightarrow}}

\begin{document}

\titulo{Trabajo Practico 2}
\submateria{Primer cuatrimestre 2014}
\materia{Algoritmos y Estructura de Datos II}
\grupo{Grupo 10}
\integrante{Luc\'ia, Parral}{162/13}{luciaparral@gmail.com}
\integrante{Nicol\'as, Roulet}{}{}
\integrante{Pablo Nicol\'as, Gomez}{}{}
\integrante{Guido Joaquin, Tamborindeguy}{}{}
\maketitle

\newpage

\tableofcontents

\newpage
\section{M\'odulo Wolfie}
\subsection{Interfaz}
%%###no tenemos no parametros formales?

  %%\InterfazFuncion{AgregarAdelante}{\Inout{l}{lista($\alpha$)}, \In{a}{$\alpha$}}{itLista($\alpha$}

%%Parámetros formales
  \subsubsection{Par\'ametros formales}
   \parbox{1.7cm}{\textbf{g\'eneros}} wolfie\\
    %%\parbox[t]{1.7cm}{\textbf{funci\'on}}\parbox[t]{\textwidth-2\parindent-1.7cm}{%
     %%\InterfazFuncion{Copiar}{\In{a}{$\alpha$}}{$\alpha$}
     %%{$res \igobs a$}
     %%[$\Theta(copy(a))$]
     %%[funci\'on de copia de $\alpha$ s]
    %%}

  \textbf{se explica con}: \tadNombre{Wolfie}.

    %%\textbf{g\'eneros}: \TipoVariable{lista$(\alpha)$}, \TipoVariable{itLista($\alpha$)}.

%%Operaciones básicas
  \subsubsection{Operaciones b\'asicas de wolfie}
 %%clientes
  \InterfazFuncion{clientes}{\In{w}{wolfie}}{itUni(cliente)}
  {$res$ \igobs crearIt(clientes($w$))}
  [$\Theta(1)$]
  [Devuelve un iterador a los clientes de un wolfie.]\\\\
 %%titulos
  \InterfazFuncion{t\'itulos}{\In{w}{wolfie}}{itUni(t\'itulo)}
  {$res$ \igobs crearItUni(t\'itulos($w$))}%
  [$\Theta(1)$]
  [Devuelve un iterador a los t\'itulos de un wolfie.]\\\\
%%promesasDe
  \InterfazFuncion{promesasDe}{\In{c}{cliente}, \In{w}{wolfie}}{itPromesa(promesa)}
  [$c \in$ clientes($w$)]
  {$res$ \igobs crearItUni(promesasDe($c$, $w$))}%
  [$\Theta(T \cdotp C \cdotp |max\_nt|)$]
  [Devuelve un iterador a las promesas de un wolfie]\\\\
  %%accionesPorCliente
  \InterfazFuncion{accionesPorCliente}{\In{c}{cliente}, \In{nt}{nombreT\'itulo}, \In{w}{wolfie}}{$nat$}%
  [$c \in$ clientes($w$) $\land$ ($\exists$ $t$:t\'itulo) ($t \in$ t\'itulos($w$) $\land$ nombre($t$) = $nt$)]
  %%###ojo con nombreTitulo
  {$res$ \igobs accionesPorCliente($c$, $nt$, $w$)}%
  [$\Theta(log(C)+|nt|)$]
  [Devuelve la cantidad de acciones que un cliente posee de un determinado t\'itulo.]\\\\
%%inaugurarWolfie
  \InterfazFuncion{inaugurarWolfie}{\In{cs}{conj(cliente)}}{wolfie}%%&&& conj(cliente)? No rompe nada usar un conjunto aca? ---lo implementamos como vector
  [$\neg\emptyset$?($cs$)]
  {$res$ \igobs inaugurarWolfie($cs$)}
  [$\Theta(\#(cs)^2)$]
  [Crea un nuevo wolfie a partir de un conjunto de clientes.]\\\\  %%&&& la complejidad habia quedado n log n si usamos merge sort
%%agregarTitulo
  \InterfazFuncion{agregarT\'itulo}{\In{t}{t\'itulo}, \Inout{w}{wolfie}}{wolfie}%
  [$w_{0}$ \igobs $w$ $\land$ ($\forall$ $t2$:t\'itulo) ($t2$ $\in$ t\'itulos($w$) $\Rightarrow$ nombre($t$) $\neq$ nombre($t2$)]
  {$w$ \igobs agregarT\'itulo($t$, $w_{0}$}%
  [$\Theta(|nombre(t)|+C)$]
%%actualizarCotizacion
  \InterfazFuncion{actualizarCotizaci\'on}{\In{nt}{nombreT\'itulo}, \In{cot}{nat}, \Inout{w}{wolfie}}{wolfie}%
  %%### ojo con NombreTítulo
  [$w_{0}$ \igobs $w$ $\land$ ($\exists$ $t$:t\'itulo) ($t$ $\in$ t\'itulos($w$) $\land$ nombre($t$) = $nt$)]
  {$w$ $\igobs$ actualizarCotizaci\'on($nt$, $cot$, $w_{0}$)}%
  [$\Theta(C \cdotp |nt|+C \cdotp log(C))$]
  [Cambia la cotizaci\'on de un determinado t\'itulo. Esta operaci\'on genera que se desencadene el cumplimiento de promesas (seg\'un corresponda): primero de venta y luego, de compra, seg\'un el orden descendente de cantidad de acciones por t\'itulo de cada cliente.]\\\\
%%agregarPromesa
  \InterfazFuncion{agregarPromesa}{\In{c}{cliente}, \In{p}{promesa}, \Inout{w}{wolfie}}{wolfie}%
  [$w_{0}$ \igobs $w$ $\land$ ($\exists$$t$: t\'itulo) ($t$ $\in$ t\'itulos($w$) $\land$ nombre($t$) = t\'itulo($p$)) $\land$ $c$ $\in$ clientes($w$) \yluego ($\forall$$p2$: promesa) ($p2$ $\in$ promesasDe($c$, $w$) $\Rightarrow$ (t\'itulo($p$) $\neq$ t\'itulo($p2$) $\lor$ tipo($p$) $\neq$ tipo($p2$)) ) $\land$ (tipo($p$) = vender $\Rightarrow$ accionesPorCliente($c$, t\'itulo($p$), $w$) $\geq$ cantidad($p$)))]
  {$w$ \igobs agregarPromesa($c$, $p$, $w_{0}$)}%
  [$\Theta(|t$\'i$tulo(p)| + log(C))$]
  [Agrega una nueva promesa.]\\\\
%%enAlza
  \InterfazFuncion{enAlza}{\In{nt}{nombreT\'itulo}, \In{w}{wolfie}}{bool}%%&&& aca estaba mal la aridad
  [($\exists$$t$: t\'itulo) ($t$ $\in$ t\'itulos($w$) $\land$ nombre($t$) = nt)]
  {$res$ \igobs enAlza($nt$, $w$)}%
  [$\Theta(|nt|)$]
  [Dado un t\'itulo, informa si est\'a o no en alza.]\\\\
  %%### ojo con NombreTítulo

%%&&& las complejidades van con O grande, no con tita

\subsection{Representaci\'on}
\subsubsection{Representaci\'on de wolfie}
%------------------------estr----------------------------
\begin{Estructura}{wolfie}[estr]
    \begin{Tupla}[estr]
      \tupItem{t\'itulos}{diccTrie(nombre, 
      	\begin{minipage}[t]{\parindent+10cm}
      		{<$arrayClientes$: array\_dimensionable(tuplaPorCliente), \\$cot$: nat, \\$enAlza$: bool, \\$maxAcc$:nat, \\$accDisponibles$: nat>),}
      	\end{minipage}}%
      	
      \tupItem{clientes}{array\_dimensionable(cliente)}
      
      \tupItem{\'ultimoLlamado}{<$cliente$: cliente, $promesas$:conj(promesa), $fueUltimo$: bool>)}%
    \end{Tupla}
    
    \begin{Tupla}[tuplaPorCliente]
    	\tupItem{cliente}{cliente,}
    	\tupItem{cantAcc}{nat,}
    	\tupItem{promCompra}{*promesa,}
    	\tupItem{promVenta}{*promesa}
    \end{Tupla}
    \indent\indent Con un orden definido por $a$<$b$ $\Leftrightarrow$ $a.cliente$ < $b.cliente$
    
    \begin{Tupla}[tuplaPorCantAcc]
    	\tupItem{cliente}{cliente,}
    	\tupItem{cantAcc}{nat,}
    	\tupItem{promCompra}{*promesa,}
    	\tupItem{promVenta}{*promesa}
    \end{Tupla}
    \indent\indent Con un orden definido por $a$<$b$ $\Leftrightarrow$ $a.cantAcc$ < $b.cantAcc$

\end{Estructura}
 %----------------------------------------------------Rep------------------------------------------
 \renewcommand{\labelenumi}{(\Roman{enumi})}
 \begin{enumerate}
 	\item Los clientes de $clientes$ son los mismos que hay dentro de $titulos$.
 	\item Las promesas de compra  son de su t\'itulo y cliente y no cumplen los requisitos para ejecutarse.
 	\item Las promesas de y venta son de su t\'itulo y cliente y no cumplen los requisitos para ejecutarse.
 	\item Las acciones disponibles de cada t\'itulo son el m\'aximo de acciones de ese t\'itulo menos la suma de las acciones de ese titulo que tengan los clientes, y son mayores o iguales a 0.
 	\item El $cliente$ de $\acute{u}ltimoLlamado$ pertenece a $clientes$
 	\item En $\acute{u}ltimoLlamado$, si $fue\acute{U}ltimo$ es true, las promesas de $promesas$ son todas las promesas que tiene $cliente$.
  \end{enumerate}
  
  \Rep[estr][e]{\\
 	(I)\begin{minipage}[t]{\linewidth-\parindent-1.3cm}
  		($\forall$ $c$: cliente) {\Large(}est\'aCliente?($c$, $e.clientes$) $\Leftrightarrow$ ($\exists$ $t$: t\'itulo) {\large(}def?($t$, $e.titulos$) \yluego est\'aCliente?($c$, obtener($t$, $e.titulos$).$arrayClientes$){\large)}{\Large)} \yluego
  	\end{minipage}\vspace{3mm}

  	(II)\begin{minipage}[t]{\linewidth-\parindent-1.3cm}
  		($\forall$ $p$: $*$promesa, $t$: nombre, $c$: cliente) {\large(}(p $\neq$ NULL $\land$ def?($t$, $e.titulos$) \yluego  est\'aCliente?($c$, obtener($t$, $e.titulos$).$arrayClientes$) \yluego buscarCliente($c$, obtener($t$, $e.titulos$).$arrayClientes$).$promCompra$=$p$) \impluego  t\'itulo($*p$)=$t$ $\land$ tipo($*p$)=compra $\land$ (l\'imite($*p$)>obtener($t$, $e.titulos$).$cot$ $\lor$ cantidad($*p$)>obtener($t$, $e.titulos$).$accDisponibles$){\large)} $\land$
  	\end{minipage}\vspace{3mm}
  	
  	(III)\begin{minipage}[t]{\linewidth-\parindent-1.3cm}
  		($\forall$ $p$: $*$promesa, $t$: nombre, $c$: cliente) {\large(}(p $\neq$ NULL $\land$ def?($t$, $e.titulos$) \yluego  est\'aCliente?($c$, obtener($t$, $e.titulos$).$arrayClientes$) \yluego buscarCliente($c$, obtener($t$, $e.titulos$).$arrayClientes$).$promVenta$=$p$) \impluego (t\'itulo($*p$)=$t$ $\land$ tipo($*p$)=venta $\land$ l\'imite($*p$) < obtener($t$, $e.titulos$).$cot$){\large)} $\land$
  	\end{minipage}\vspace{3mm}
  	
  	(IV)\begin{minipage}[t]{\linewidth-\parindent-2cm}
		($\forall$ $nt$: nombreT) (def?($nt$, $e.titulos$) \impluego (obtener($nt$, $e.titulos$).$accDisponibles$ = obtener($nt$, $e.titulos$).$maxAcc$ - sumaAccClientes(obtener($nt$, $e.titulos$).$arrayClientes$) $\land$ obtener($nt$, $e.titulos$).$accDisponibles$ $\geq$ 0)) $\land$
  	\end{minipage}\vspace{3mm}
  	
  	(V)\begin{minipage}[t]{\linewidth-\parindent-2cm}
		(est\'aCliente($e.\acute{u}ltimoLlamado.cliente$, $e.clientes$)) $\land$
  	\end{minipage}\vspace{3mm}
  	
	(VI)\begin{minipage}[t]{\linewidth-\parindent-2cm}
		($e.\acute{u}ltimoLlamado.fue\acute{U}ltimo$ $\Rightarrow$ ($\forall$ $p$: promesa) {\Large(}pertenece?($p$, $e.\acute{u}ltimoLlamado.promesas$) $\Leftrightarrow$ {\large(}def?(t\'itulo($p$), $e.titulos$) \yluego \\ {\IF tipo($p$)=compra THEN buscarCliente($e.\acute{u}ltimoLlamado.cliente$, obtener(t\'itulo($p$), $e.titulos$).$arrayClientes$).promCompra = $p$ ELSE buscarCliente($e.\acute{u}ltimoLlamado.cliente$, obtener(t\'itulo($p$), $e.titulos$).$arrayClientes$).promVenta = $p$ FI}{\large)}{\Large)}
  	\end{minipage}\vspace{3mm}
  	
  	
  }\mbox{}
% para el rep, necesito las operaciones: def? y obtener de dicTrie, estaCliente? y buscarCliente de arrays de tuplas con cliente, sumaAccClientes 



\section{M\'odulo DiccionarioTrie(alpha)}
\subsection{Interfaz}

%%Parámetros formales
  \subsubsection{Par\'ametros formales}
   \parbox{1.7cm}{\textbf{g\'eneros}} string, $\alpha$\\

  \textbf{se explica con}: \tadNombre{Diccionario(string, $\alpha$)}, \tadNombre{Iterador Unidireccional}.

  \textbf{g\'eneros}: \TipoVariable{diccString($\alpha$), itDicc(diccString)}. %%###revisar

%%Operaciones básicas
  \subsubsection{Operaciones b\'asicas de Diccionario String($\alpha$)}
%%crearDicc
  \InterfazFuncion{CrearDicc}{}{diccString($\alpha$)}
  {$res \igobs vacio$}%
  [O$(1)$]
  [Crea un diccionario vac\'io.]\\\\
%%definir
  \InterfazFuncion{Definir}{\Inout{d}{diccString($\alpha$)}, \In{c}{string}, \In{s}{$\alpha$}}{}
  [$d \igobs d_0 \wedge \neg def?(d,c)$]
  {$d \igobs definir(d_0,c,s)$}%
  [O$(longitud(c))$]
  [Define la clave c con el significado s en el diccionario d.]\\\\
%%definido?  
  \InterfazFuncion{Definido?}{\In{d}{diccString($\alpha$)}, \In{c}{string}}{bool}
  {$res \igobs def?(c,d)$}%
  [O$(longitud(c))$]
  [Devuelve true si y solo si c est\'a definido como clave en el diccionario.]\\\\
%%significado  
  \InterfazFuncion{Significado}{\In{d}{diccString($\alpha$)}, \In{c}{string}}{$\alpha$}
  [$def?(c,d)$]
  {$res \igobs obtener(c,d)$}%
  [O$(longitud(c))$]
  [Devuelve el significado con clave c.]
  [No se devuelve una copia del $\alpha$ en res, se devuelve una referencia a la original.]\\\\

 \subsubsection{Operaciones b\'asicas del iterador de claves de Diccionario String($\alpha$)}
  \InterfazFuncion{CrearIt}{\In{d}{diccString($\alpha$)}}{itClaves($string$)}
  {res \igobs crearIt(d.claves) }
  [O$(1)$]
  [Crea y devuelve un iterador de claves Diccionario String.]\\\\

  \InterfazFuncion{HayMas?}{\In{d}{itClaves($string$)}}{$bool$}
  {res \igobs hayMas?($it$)}%
  [O$(longitud(c))$] %%##revisar
  [Informa si hay m\'as elementos por iterar.]\\\\

  \InterfazFuncion{Actual}{\In{d}{itClaves($string$)}}{$string$}
  {res \igobs actual($it$)}
  [O$(longitud(c))$]%%##revisar
  [Devuelve la clave de la posici\'on actual.]\\\\

  \InterfazFuncion{Avanzar}{\Inout{it}{itClaves($string$)}}{itClaves($\alpha$)}
  [hayMas?($it$) $\land$ $it = it_{0}$] %%##revisar
  {res \igobs avanzar($it_{0}$)}
  [O$(longitud(c))$]%%##revisar
  [Avanza a la pr\'oxima clave.]\\\\


\subsection{Representacion}
  
  \subsubsection{Representaci\'on del Diccionario String($\alpha$)}
%%Estructura
  \begin{Estructura}{diccString$(\alpha)$}[estrDic]
    \begin{Tupla}[estrDic]
      \tupItem{raiz}{puntero(nodo)}
      \tupItem{claves}{lista\_enlazada($string$)}%%
    \end{Tupla}
\end{Estructura}

  \begin{Estructura}{Nodo}[estrNodo]
    \begin{Tupla}[estrNodo]
      \tupItem{valor}{puntero($\alpha$)}
      \tupItem{hijos}{arreglo\_estatico[256](puntero(nodo))}
    \end{Tupla}
\end{Estructura}

%%Rep 

\renewcommand{\labelenumi}{(\Roman{enumi})}
 \begin{enumerate}
 	\item Existe un \'unico camino entre cada nodo y el nodo raiz (es decir, no hay ciclos).
 	\item Todos los nodos hojas, es decir, todos los que tienen su arreglo hijos con todas sus posiciones en NULL, tienen que tener un valor distinto de NULL.
 	\item Raiz es distinto de NULL
 	\item En claves est\'a el camino que se recorre desde la raiz hasta cada nodo hoja.
  \end{enumerate}

\Rep[estrDic][e]{\\
 	\begin{minipage}[t]{\linewidth-\parindent-1.3cm}
  	{raiz != NULL \yluego \, noHayCiclos(e) $\land$ todasLasHojasTienenValor(e) $\land$\\ hayHojas(e) $\Rightarrow$ |e.claves|> 0 $\land$ \\($\forall$ c $\in$ caminosANodos(e))($\exists$ i \{0..|e.claves|\}) e.claves[i] = c}
  	\end{minipage}\vspace{3mm}
  	 }\mbox{}

 \Abs[estrDicc]{dicc(string,$\alpha$)}[e]{$d$}{($\forall$ c:string)(definido?($c,d$)) = ($\exists$ n: nodo)(n $\in$ todasLasHojas($e$)) n.valor != NULL \\ $\land$ ($\exists$ i \{0..|e.claves|\}) e.claves[i] = c $\yluego$ significado($c,d$) = leer($e.clave$).valor}


 \subsubsection{Operaciones auxiliares del invatriante de Representaci\'on}

  \tadOperacion{noHayCiclos}{puntero(nodo)}{bool}{}
  \tadAxioma{noHayCiclos($n,p$)}{($\exists$ n:nat)(($\forall$ c: string)(|s| = n $\Rightarrow$ leer($p,s$) = NULL))}
  \tadOperacion{leer}{puntero(nodo), string}{bool}{}
  \tadAxioma{leer($p,s$)}{\IF vacia?($s$) THEN p $\rightarrow$ valor ELSE {\IF p $\rightarrow$ hijos[prim(s)] = NULL THEN NULL ELSE leer(p $\rightarrow$ hijos[prim(s)], fin(s)) FI} FI}
  \tadOperacion{todosNull}{arreglo(puntero(nodo))}{bool}{}
  \tadAxioma{todosNull($a$)}{auxTodosNull($a, 0$)}
  \tadOperacion{auxTodosNull}{arreglo(puntero(nodo)), nat}{bool}{}
  \tadAxioma{auxTodosNull($a, i$)}{\IF i < |a| THEN a[i] == NULL $\land$ auxTodosNull($a, i+1$) ELSE a[i].valor == NULL FI}
  \tadOperacion{esHoja}{puntero(nodo)}{bool}{}
  \tadAxioma{esHoja($p$)}{\IF p == NULL THEN false ELSE todosNull(p.hijos) FI}
  \tadOperacion{todasLasHojas}{puntero(nodo), nat}{conj(nodo)}{}
  \tadAxioma{todasLasHojas($p, n$)}{\IF p == NULL THEN false ELSE {\IF esHoja($p$) THEN Ag(*p, vacio) ELSE auxTodasLasHojas((*p).hijos, 256) FI} FI}

  \tadOperacion{auxTodasLasHojas}{arreglo(puntero(nodo)), nat}{conj(nodo)}{}
  \tadAxioma{auxTodasLasHojas($a, n$)}{hojasDeHijos($a,n,0$)}

  \tadOperacion{hojasDeHijos}{arreglo(puntero(nodo)), nat, nat}{conj(nodo)}{}
  \tadAxioma{hojasDeHijos($a, n, i$)}{\IF i = n THEN $\emptyset$ ELSE todasLasHojas(a[i]) $\cup$ hojasDeHijos($a, n, (i+1)$) FI}

  \tadOperacion{todasLasHojasTienenValor}{puntero(nodo)}{bool}{}
  \tadAxioma{todasLasHojasTienenValor($p$)}{auxTodasLasHojasTienenValor(todasLasHojas($p, 256$))}

  \tadOperacion{auxTodasLasHojasTienenValor}{arreglo(puntero(nodo))}{bool}{}
  \tadAxioma{auxTodasLasHojasTienenValor($a$)}{\IF |a| = 0 THEN true ELSE dameUno(a).valor != NULL $\land$ auxTodasLasHojasTienenValor(sinUno(a)) FI}

  \subsubsection{Representaci\'on del iterador de Claves del Diccionario String($\alpha$)}

    \begin{Estructura}{itClaves($string$)}[puntero(nodo)]
    \end{Estructura}

    Su Rep y Abs son los de itSecu($\alpha$) definido en el apunte de iteradores..

  \subsection{Algoritmos}
\subsubsection{Algoritmos de Diccionario String}

\textsc{iCrearDicc}() $\rightarrow$ \textbf{res} = diccString($\alpha$)
\begin{lstlisting}[mathescape]
 n $\leftarrow$ nodo
 n $\leftarrow$ crearNodo()
 raiz $\leftarrow$ *n
\end{lstlisting}
\textbf{Complejidad}\\

\textsc{iCrearNodo}() $\rightarrow$ \textbf{res} = nodo
\begin{lstlisting}[mathescape]
 d : arreglo\_estatico[256]
 i $\leftarrow$ 0
 while (i < 256)
 	d[i] $\leftarrow$ NULL
 endWhile
 hijos $\leftarrow$ d
 valor $\leftarrow$ NULL
\end{lstlisting}
\textbf{Complejidad}\\

\textsc{iDefinir}(\textbf{in/out} diccString($\alpha$): d, \textbf{in} string: c, \textbf{in} alfa: s)
\begin{lstlisting}[mathescape]
 i $\leftarrow$ 0
 p $\leftarrow$ d.raiz
 while (i < (longitud(s)))
 	if (p.hijos[ord(s[i])] == NULL)
		n: nodo $\leftarrow$ crearNodo()
		p.hijos[ord(s[i])] $\leftarrow$ *n
	endIf
 p $\leftarrow$ p.hijos[ord(s[i])]
 i++
 endWhile
 p.valor $\leftarrow$ a
 agregarAdelante(hijos, c)
\end{lstlisting}
\textbf{Complejidad}\\

\textsc{iSignificado}(\textbf{in} diccString($\alpha$): d, \textbf{in} string: c) $\rightarrow$ \textbf{res} = $\alpha$
\begin{lstlisting}[mathescape]
 i $\leftarrow$ 0
 p $\leftarrow$ d.raiz
 while (i < (longitud(s)))
	p $\leftarrow$ p.hijos[ord(s[i])]
 i++
 endWhile
 return p.valor
\end{lstlisting}
\textbf{Complejidad}\\

\textsc{iDefinido?}(\textbf{in} diccString($\alpha$): d, \textbf{in} string: c) $\rightarrow$ \textbf{res} = bool
\begin{lstlisting}[mathescape]
 i $\leftarrow$ 0
 p $\leftarrow$ d.raiz
 while (i < (longitud(s)))
 	if (p.hijos[ord(s[i])] != NULL)
		p $\leftarrow$ p.hijos[ord(s[i])]
		i++
	else
		return false
	endIf
 endWhile
 return p.valor != NULL
\end{lstlisting}
\textbf{Complejidad}\\

\textsc{iClaves}(\textbf{in} diccString($\alpha$): d) $\rightarrow$ \textbf{res} = lista\_enlazada(string)
\begin{lstlisting}[mathescape]
return itClaves(d)
\end{lstlisting}
\textbf{Complejidad}\\

\subsubsection{Algoritmos del iterador de claves del Diccionario String}

Utiliza los mismos algoritmos que itSecu($\alpha$) definido en el apunte de iteradores.


%%\section{M\'odulo Wolfie}
%%\subsection{Interfaz}
%%###no tenemos no parametros formales?

  %%\InterfazFuncion{AgregarAdelante}{\Inout{l}{lista($\alpha$)}, \In{a}{$\alpha$}}{itLista($\alpha$}

%%Parámetros formales
  \subsubsection{Par\'ametros formales}
   \parbox{1.7cm}{\textbf{g\'eneros}} wolfie\\
    %%\parbox[t]{1.7cm}{\textbf{funci\'on}}\parbox[t]{\textwidth-2\parindent-1.7cm}{%
     %%\InterfazFuncion{Copiar}{\In{a}{$\alpha$}}{$\alpha$}
     %%{$res \igobs a$}
     %%[$\Theta(copy(a))$]
     %%[funci\'on de copia de $\alpha$ s]
    %%}

  \textbf{se explica con}: \tadNombre{Wolfie}.

    %%\textbf{g\'eneros}: \TipoVariable{lista$(\alpha)$}, \TipoVariable{itLista($\alpha$)}.

%%Operaciones básicas
  \subsubsection{Operaciones b\'asicas de wolfie}
 %%clientes
  \InterfazFuncion{clientes}{\In{w}{wolfie}}{itUni(cliente)}
  {$res$ \igobs crearIt(clientes($w$))}
  [$\Theta(1)$]
  [Devuelve un iterador a los clientes de un wolfie.]\\\\
 %%titulos
  \InterfazFuncion{t\'itulos}{\In{w}{wolfie}}{itUni(t\'itulo)}
  {$res$ \igobs crearItUni(t\'itulos($w$))}%
  [$\Theta(1)$]
  [Devuelve un iterador a los t\'itulos de un wolfie.]\\\\
%%promesasDe
  \InterfazFuncion{promesasDe}{\In{c}{cliente}, \In{w}{wolfie}}{itPromesa(promesa)}
  [$c \in$ clientes($w$)]
  {$res$ \igobs crearItUni(promesasDe($c$, $w$))}%
  [$\Theta(T \cdotp C \cdotp |max\_nt|)$]
  [Devuelve un iterador a las promesas de un wolfie]\\\\
  %%accionesPorCliente
  \InterfazFuncion{accionesPorCliente}{\In{c}{cliente}, \In{nt}{nombreT\'itulo}, \In{w}{wolfie}}{$nat$}%
  [$c \in$ clientes($w$) $\land$ ($\exists$ $t$:t\'itulo) ($t \in$ t\'itulos($w$) $\land$ nombre($t$) = $nt$)]
  %%###ojo con nombreTitulo
  {$res$ \igobs accionesPorCliente($c$, $nt$, $w$)}%
  [$\Theta(log(C)+|nt|)$]
  [Devuelve la cantidad de acciones que un cliente posee de un determinado t\'itulo.]\\\\
%%inaugurarWolfie
  \InterfazFuncion{inaugurarWolfie}{\In{cs}{conj(cliente)}}{wolfie}%%&&& conj(cliente)? No rompe nada usar un conjunto aca? ---lo implementamos como vector
  [$\neg\emptyset$?($cs$)]
  {$res$ \igobs inaugurarWolfie($cs$)}
  [$\Theta(\#(cs)^2)$]
  [Crea un nuevo wolfie a partir de un conjunto de clientes.]\\\\  %%&&& la complejidad habia quedado n log n si usamos merge sort
%%agregarTitulo
  \InterfazFuncion{agregarT\'itulo}{\In{t}{t\'itulo}, \Inout{w}{wolfie}}{wolfie}%
  [$w_{0}$ \igobs $w$ $\land$ ($\forall$ $t2$:t\'itulo) ($t2$ $\in$ t\'itulos($w$) $\Rightarrow$ nombre($t$) $\neq$ nombre($t2$)]
  {$w$ \igobs agregarT\'itulo($t$, $w_{0}$}%
  [$\Theta(|nombre(t)|+C)$]
%%actualizarCotizacion
  \InterfazFuncion{actualizarCotizaci\'on}{\In{nt}{nombreT\'itulo}, \In{cot}{nat}, \Inout{w}{wolfie}}{wolfie}%
  %%### ojo con NombreTítulo
  [$w_{0}$ \igobs $w$ $\land$ ($\exists$ $t$:t\'itulo) ($t$ $\in$ t\'itulos($w$) $\land$ nombre($t$) = $nt$)]
  {$w$ $\igobs$ actualizarCotizaci\'on($nt$, $cot$, $w_{0}$)}%
  [$\Theta(C \cdotp |nt|+C \cdotp log(C))$]
  [Cambia la cotizaci\'on de un determinado t\'itulo. Esta operaci\'on genera que se desencadene el cumplimiento de promesas (seg\'un corresponda): primero de venta y luego, de compra, seg\'un el orden descendente de cantidad de acciones por t\'itulo de cada cliente.]\\\\
%%agregarPromesa
  \InterfazFuncion{agregarPromesa}{\In{c}{cliente}, \In{p}{promesa}, \Inout{w}{wolfie}}{wolfie}%
  [$w_{0}$ \igobs $w$ $\land$ ($\exists$$t$: t\'itulo) ($t$ $\in$ t\'itulos($w$) $\land$ nombre($t$) = t\'itulo($p$)) $\land$ $c$ $\in$ clientes($w$) \yluego ($\forall$$p2$: promesa) ($p2$ $\in$ promesasDe($c$, $w$) $\Rightarrow$ (t\'itulo($p$) $\neq$ t\'itulo($p2$) $\lor$ tipo($p$) $\neq$ tipo($p2$)) ) $\land$ (tipo($p$) = vender $\Rightarrow$ accionesPorCliente($c$, t\'itulo($p$), $w$) $\geq$ cantidad($p$)))]
  {$w$ \igobs agregarPromesa($c$, $p$, $w_{0}$)}%
  [$\Theta(|t$\'i$tulo(p)| + log(C))$]
  [Agrega una nueva promesa.]\\\\
%%enAlza
  \InterfazFuncion{enAlza}{\In{nt}{nombreT\'itulo}, \In{w}{wolfie}}{bool}%%&&& aca estaba mal la aridad
  [($\exists$$t$: t\'itulo) ($t$ $\in$ t\'itulos($w$) $\land$ nombre($t$) = nt)]
  {$res$ \igobs enAlza($nt$, $w$)}%
  [$\Theta(|nt|)$]
  [Dado un t\'itulo, informa si est\'a o no en alza.]\\\\
  %%### ojo con NombreTítulo

%%&&& las complejidades van con O grande, no con tita

\subsection{Representaci\'on}
\subsubsection{Representaci\'on de wolfie}
%------------------------estr----------------------------
\begin{Estructura}{wolfie}[estr]
    \begin{Tupla}[estr]
      \tupItem{t\'itulos}{diccTrie(nombre, 
      	\begin{minipage}[t]{\parindent+10cm}
      		{<$arrayClientes$: array\_dimensionable(tuplaPorCliente), \\$cot$: nat, \\$enAlza$: bool, \\$maxAcc$:nat, \\$accDisponibles$: nat>),}
      	\end{minipage}}%
      	
      \tupItem{clientes}{array\_dimensionable(cliente)}
      
      \tupItem{\'ultimoLlamado}{<$cliente$: cliente, $promesas$:conj(promesa), $fueUltimo$: bool>)}%
    \end{Tupla}
    
    \begin{Tupla}[tuplaPorCliente]
    	\tupItem{cliente}{cliente,}
    	\tupItem{cantAcc}{nat,}
    	\tupItem{promCompra}{*promesa,}
    	\tupItem{promVenta}{*promesa}
    \end{Tupla}
    \indent\indent Con un orden definido por $a$<$b$ $\Leftrightarrow$ $a.cliente$ < $b.cliente$
    
    \begin{Tupla}[tuplaPorCantAcc]
    	\tupItem{cliente}{cliente,}
    	\tupItem{cantAcc}{nat,}
    	\tupItem{promCompra}{*promesa,}
    	\tupItem{promVenta}{*promesa}
    \end{Tupla}
    \indent\indent Con un orden definido por $a$<$b$ $\Leftrightarrow$ $a.cantAcc$ < $b.cantAcc$

\end{Estructura}
 %----------------------------------------------------Rep------------------------------------------
 \renewcommand{\labelenumi}{(\Roman{enumi})}
 \begin{enumerate}
 	\item Los clientes de $clientes$ son los mismos que hay dentro de $titulos$.
 	\item Las promesas de compra  son de su t\'itulo y cliente y no cumplen los requisitos para ejecutarse.
 	\item Las promesas de y venta son de su t\'itulo y cliente y no cumplen los requisitos para ejecutarse.
 	\item Las acciones disponibles de cada t\'itulo son el m\'aximo de acciones de ese t\'itulo menos la suma de las acciones de ese titulo que tengan los clientes, y son mayores o iguales a 0.
 	\item El $cliente$ de $\acute{u}ltimoLlamado$ pertenece a $clientes$
 	\item En $\acute{u}ltimoLlamado$, si $fue\acute{U}ltimo$ es true, las promesas de $promesas$ son todas las promesas que tiene $cliente$.
  \end{enumerate}
  
  \Rep[estr][e]{\\
 	(I)\begin{minipage}[t]{\linewidth-\parindent-1.3cm}
  		($\forall$ $c$: cliente) {\Large(}est\'aCliente?($c$, $e.clientes$) $\Leftrightarrow$ ($\exists$ $t$: t\'itulo) {\large(}def?($t$, $e.titulos$) \yluego est\'aCliente?($c$, obtener($t$, $e.titulos$).$arrayClientes$){\large)}{\Large)} \yluego
  	\end{minipage}\vspace{3mm}

  	(II)\begin{minipage}[t]{\linewidth-\parindent-1.3cm}
  		($\forall$ $p$: $*$promesa, $t$: nombre, $c$: cliente) {\large(}(p $\neq$ NULL $\land$ def?($t$, $e.titulos$) \yluego  est\'aCliente?($c$, obtener($t$, $e.titulos$).$arrayClientes$) \yluego buscarCliente($c$, obtener($t$, $e.titulos$).$arrayClientes$).$promCompra$=$p$) \impluego  t\'itulo($*p$)=$t$ $\land$ tipo($*p$)=compra $\land$ (l\'imite($*p$)>obtener($t$, $e.titulos$).$cot$ $\lor$ cantidad($*p$)>obtener($t$, $e.titulos$).$accDisponibles$){\large)} $\land$
  	\end{minipage}\vspace{3mm}
  	
  	(III)\begin{minipage}[t]{\linewidth-\parindent-1.3cm}
  		($\forall$ $p$: $*$promesa, $t$: nombre, $c$: cliente) {\large(}(p $\neq$ NULL $\land$ def?($t$, $e.titulos$) \yluego  est\'aCliente?($c$, obtener($t$, $e.titulos$).$arrayClientes$) \yluego buscarCliente($c$, obtener($t$, $e.titulos$).$arrayClientes$).$promVenta$=$p$) \impluego (t\'itulo($*p$)=$t$ $\land$ tipo($*p$)=venta $\land$ l\'imite($*p$) < obtener($t$, $e.titulos$).$cot$){\large)} $\land$
  	\end{minipage}\vspace{3mm}
  	
  	(IV)\begin{minipage}[t]{\linewidth-\parindent-2cm}
		($\forall$ $nt$: nombreT) (def?($nt$, $e.titulos$) \impluego (obtener($nt$, $e.titulos$).$accDisponibles$ = obtener($nt$, $e.titulos$).$maxAcc$ - sumaAccClientes(obtener($nt$, $e.titulos$).$arrayClientes$) $\land$ obtener($nt$, $e.titulos$).$accDisponibles$ $\geq$ 0)) $\land$
  	\end{minipage}\vspace{3mm}
  	
  	(V)\begin{minipage}[t]{\linewidth-\parindent-2cm}
		(est\'aCliente($e.\acute{u}ltimoLlamado.cliente$, $e.clientes$)) $\land$
  	\end{minipage}\vspace{3mm}
  	
	(VI)\begin{minipage}[t]{\linewidth-\parindent-2cm}
		($e.\acute{u}ltimoLlamado.fue\acute{U}ltimo$ $\Rightarrow$ ($\forall$ $p$: promesa) {\Large(}pertenece?($p$, $e.\acute{u}ltimoLlamado.promesas$) $\Leftrightarrow$ {\large(}def?(t\'itulo($p$), $e.titulos$) \yluego \\ {\IF tipo($p$)=compra THEN buscarCliente($e.\acute{u}ltimoLlamado.cliente$, obtener(t\'itulo($p$), $e.titulos$).$arrayClientes$).promCompra = $p$ ELSE buscarCliente($e.\acute{u}ltimoLlamado.cliente$, obtener(t\'itulo($p$), $e.titulos$).$arrayClientes$).promVenta = $p$ FI}{\large)}{\Large)}
  	\end{minipage}\vspace{3mm}
  	
  	
  }\mbox{}
% para el rep, necesito las operaciones: def? y obtener de dicTrie, estaCliente? y buscarCliente de arrays de tuplas con cliente, sumaAccClientes 




%\newpage
%\section{Modulo2}
%\input{modulo_dictrie.tex}

\end{document}
